\documentclass[a4j,fleqn,dvipdfmx,twocolumn]{jsarticle}
\usepackage[left=20truemm,right=20truemm,top=20truemm,bottom=25truemm]{geometry}
\usepackage{amsmath}
\begin{document}
\section{使い方}
要するに著者は特に気にせず通常通り\verb+\cite{aaa,bbb}+と
入力すればOKです.

ここでは参考文献リストの動作を確認します.
学会のテンプレートで参照されている文献\cite{1,2,3,4,5,6}は,
左記のように表示されますが,学会のテンプレート側で上付きの${}^{1)}, {}^{2)}$といった
処理がされます.${}^{1),2)}$や${}^{1\text{~}6)}$といった表記も
学会のテンプレート側でやってもらえます.

\section{注意点}
\subsection{upbibtexを使ってください}
bibtexuを使うとカンマやピリオド,スペースなどの処理がupbibtexと違うようで
出力が乱れます.upbibtexをお使いください.

ちなみに,SICE会誌本体のtexファイルは
\begin{enumerate}
	\item \verb+\documentclass[uplatex]{jsice}+としてuplatexでコンパイル
	\item platexで(``u''のないコマンドで)コンパイル
\end{enumerate}
とするとコンパイルが通りました.

\subsection{日本名の記述の流儀にあわせて,設定を変更してください}
オリジナルのjecon.bstから継承している設定項目です.
\begin{enumerate}
	\item bibファイルで\verb+author = {姓, 名}+ あるいは \verb+{名 姓}+ 
	と記入している(つまり英語名と同じルールで記入している)著者:
	特に設定は必要ありません.
	\item bibファイルで\verb+author = {姓 名}+ あるいは \verb+{名, 姓}+
	と記入している著者:
	bstファイルの\verb+FUNCTION {bst.sei.mei.order}+の箇所を書き換えてください.
	bibファイル自体の修正は必要ありません.
\end{enumerate}

\subsection{bibファイルでのページ数の記入の流儀にあわせて,設定を変更してください}
SICEの雑誌に合わせて``aaa/bbb''といった形式にする処理を行っています.このため,
\begin{enumerate}
	\item bibファイルで\verb+pages = {aaa--bbb}+とダッシュ2回で記入している著者:
	      特に設定は必要ありません.
	\item bibファイルで\verb+pages = {aaa-bbb}+とダッシュ1回で記入している著者:
	      \verb+FUNCTION {bst.slashfysingledash}+の箇所を修正してご利用ください.
	      bibファイル自体の修正は必要ありません.
	\item bibファイルで\verb+pages = {aaa - bbb}+や
	      \verb+pages = {aaa -- bbb}+と空白を入れて記入している著者:
	      スペースを取る処理は現状bstファイルにはありませんので,
	      \begin{enumerate}
		      \item bibファイルを修正する
		      \item 吐き出されたbblファイルを修正する
	      \end{enumerate}
	      などで修正をご検討ください.
\end{enumerate}
% 参考文献リスト
\bibliographystyle{jecon_sice}
\bibliography{sice_example}
\end{document}
